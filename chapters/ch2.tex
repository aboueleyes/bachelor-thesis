\chapter{Temporal Representation and Reasoning}
\label{ch:states-and-events}
In this chapter, we will take a look at previous work on temporal representation and reasoning.


\section{Vendler's Categorization}

In his paper \cite{vendler1957verbs}, Vendler proposes a categorization of verbs based on their temporal and aspectual properties. The goal of this categorization is to provide a framework for understanding how verbs relate to time and how their meanings are structured.

Verbs are a central component of natural language and are used to express actions, states, and events. They are also crucial for representing the temporal structure of sentences, as they allow us to talk about past, present, and future events. However, not all verbs are the same when it comes to their temporal properties. Some verbs, like ``run'' or ``sing'', describe actions that occur over a period of time, while others, like ``know'' or ``believe'', describe states or attitudes that hold at a particular point in time.


Vendler proposes a categorization of verbs based on their temporal and aspectual properties. He identifies four main categories of verbs: \textit{activities}, \textit{accomplishments}, \textit{achievements}, and \textit{states}.

\textbf{Activities} are verbs that describe ongoing processes or actions that have no inherent endpoint. Examples of activities include ``run'', ``swim'', and ``write''.


\textbf{Accomplishments} are verbs that describe actions that have a clear endpoint or goal. Examples of accomplishments include ``build'', ``solve'', and ``paint''.


\textbf{Achievements} are verbs that describe events that happen suddenly or instantaneously, without any inherent duration. Examples of achievements include ``win'', ``die'', and ``arrive''.


\textbf{States} are verbs that describe conditions or states of being that hold over a period of time. Examples of states include ``know'', ``believe'', and ``love''.

Each of these categories has a different temporal and aspectual structure, and Vendler argues that these differences are crucial for understanding the meaning of verbs and their relationships to time.



\begin{exmp} Analysis of the verb ``think''
	\begin{enumerate}
		\item He is ``thinking'' about John. \label{first}
		\item He ``thinks'' that John is a philosopher. \label{second}
	\end{enumerate}

	The verb ``think'' functions differently in the two sentences presented. In the first sentence \ref{first}, ``thinking'' is characterized as a \textit{process} that involves ongoing mental activity related to John, whereas in the second sentence \ref{second}, ``thinks'' describes a \textit{state} of belief or opinion about John.

	The first sentence \ref{first} suggests that the activity of ``thinking'' about something is a deliberate and continuous process that occurs over time. The phrase ``deliberately'' highlights the intentional nature of the activity. Furthermore, if one were to think about John for a period of time, it would imply that the thinking was consistent and continuous throughout that time period. This notion is expressed as \textit{homogeneity}.
\end{exmp}


\section{Interval Algebra}

Allen's paper presents a framework called the \textit{Interval Algebra} for representing and reasoning about actions and time in natural language processing and artificial intelligence systems \cite{allen1984towards}. The Interval Algebra uses a set of basic temporal relations to describe the temporal and causal relationships between events and actions. For example, it can represent relations such as ``before'',``after'',``meets'',``overlaps'',``during'', and ``equals'' between different intervals of time.

Allen argues that the Interval Algebra can be used to model a wide range of natural language expressions involving \textit{actions} and \textit{events}. This includes simple statements about temporal order, such as ``John ate breakfast before he went to work'', as well as more complex descriptions of causality and concurrency, such as ``When John opened the door, the alarm went off.''

Allen presents a \textit{many-sorted} \textit{first-order} logic, defining several sorts related to various ontological entities identified.

Allen takes the \textit{interval} as the temporal primitive, completely excluding the notion of \textit{point} from the theory, arguing that our direct experience
is with events and actions which typically take time to occur.

He Introduced a set of 13 mutually exclusive binary relations between intervals and
defined the \textit{interval algebra} as all the possible disjunctions that can be formed from these relations.

In this framework, time is modeled as \textit{linear} and \textit{infinite}, the interval structure implies time is \textit{continuous}.

\begin{exmp} Allen's Interval Algebra

	\begin{enumerate}
		\item DURING($t_1$, $t_2$) time interval $t_1$ is fully contained within time interval $t_2$.

		\item STARTS($t_1$, $t_2$) time interval $t_1$ shares a starting point with time interval $t_2$ but ends before $t_2$.

		\item FINISHES($t_1$, $t_2$) time interval $t_1$ shares the same end as time interval $t_2$ but starts after $t_2$.

		\item IN($t_1$, $t_2$) $\iff$ (DURING($t_1$, $t_2$) $\lor$ STARTS($t_1$, $t_2$) $\lor$ FINISHES($t_1$, $t_2$)).

		\item HOLDS(\(\phi\), \(t\)) is true iff \(\phi\) holds during \(t\).
	\end{enumerate}



	\begin{center}
		HOLDS(\(\phi\), \(T\)) \(\iff ( \forall t \) IN $ (t,T) \implies $ HOLDS($\phi, t$)).
	\end{center}

	This definition states that HOLDS($\phi$, $T$) is true iff $\phi$ holds during every subinterval $t$ of $T$. In other words, $\phi$ holds throughout the entire interval $T$.
\end{exmp}

\begin{exmp} Complex logical expressions

	To allow properties to name complex logical expressions, there is a set of
	functions \textit{and}, \textit{or}, \textit{not}, \textit{all}, \textit{and} \textit{exists}, that correspond to the logical operators
	$\&, \lor, \sim, \forall$ and $\exists$ respectively.
	Conjunction moves through the HOLDS predicate freely:
	\begin{equation}
		\text{HOLDS}(\text{and}(p, q), t) \iff \text{HOLDS}(p, t) \  \& \  \text{HOLDS}(q, t)
	\end{equation}

	Negation is defined as follows:
	\begin{equation}
		\text{HOLDS}(\text{not}(p), T) \iff (\forall t.\text{IN}(t, T) \implies \sim \text{HOLDS}(p, t))
	\end{equation}
\end{exmp}




\subsection{Occurrences}
Allen uses three basic entities that are associated with time which are: \textit{properties}, \textit{events}, and \textit{processes}.
\begin{enumerate}
	\item \textbf{Events} describe an activity that involves an outcome. Examples of events such as ``John walked to the store''.

	\item \textbf{Processes} refer to some activity not associated with a result. Examples of processes include ``John was walking'' and ``John was running''.

	\item \textbf{Properties} are conditions that hold over a period of time. Examples of properties include ``John owns a cat''.
\end{enumerate}
One way to distinguish between \textit{events} and \textit{processes} is that one can count how many times an event occurs, but one cannot count
the number of times a process occurs.


The predicate OCCUR takes an event and a time interval and is true if the event happens over the time interval $t$ and there is no subinterval of $t$
over which the event occurs.

\begin{center}
	OCCUR($e$, $t$) $\land$ IN($t^\prime$, $t$) $\implies$ \(\lnot\) OCCUR($e$, $t^\prime$).
\end{center}

Another way is to consider the characteristics of the set of temporal intervals that they hold or occur over.
Consider the \textit{process} ``I am walking'' over the Interval $I$. Unlike \textit{events}, this process might be occurring over subintervals of $I$ however a \textit{property} must be holding over every subinterval of $I$.


If a process $p$ took place over interval $t$ is denoted by the formula OCCURRING$(p,t)$.


As we illustrated, a process occurring over an interval $T$, must be occurring over at least one subinterval of $T$.

\section{Shoham Temporal Logic}

In his paper \cite{shoham1988temporal}.
Shoham shows how Allen' Ontology have a problem defining the properties as they are not only not precise but they look like the $\mathcal{FOPC}$, so he missed the-off-shelf $\mathcal{FOPC}$.

Also he shows that Allen's Ontology introduces unnecessary complexity just because Allen went for using intervals instead of time points.

Shoham also the distinction between \textit{properties}, \textit{events}, and \textit{processes} is unnecessary and it's not very useful.


\subsection{Shoham's Interval Logic}
\subsubsection{Syntax}
\begin{itemize}
	\item Given $TC$ a set of time points symbols,
	\item $C$ a set of constant symbols that's disjoint from $TC$;
	\item $TV$ a set of temproal variables,
	\item $V$ a set of variables that's disjoint from $TV$.
	\item $TF$ a set of temporal function symbols,
	\item $F$ a set of function symbols that's disjoint from $TF$.
	\item $R$ a set of relation symbols.
\end{itemize}

The set of temporal formulas is defined inductively as follows:
\begin{enumerate}
	\item All members of $TC$ are temporal terms.
	\item All members of $TV$ are temporal terms.
	\item if $trm_1, \dots, trm_n$ are temporal terms, and $f \in TF$ is an $n$-ary temporal function symbol, then $f(trm_1, \dots, trm_n)$ is a temporal term.
\end{enumerate}

The set of \textit{nontemporal formulas} is defined inductively in exactly the same way as the set of temporal formulas, with
$TC$ replace by $C$, $TV$ replaced by $V$, $TF$ replaced by $F$.

The set of well-formed formulas (wffs) is defined inductively as follows:
\begin{enumerate}
	\item if $trm_a$ and $trm_b$ are temporal terms, then $trm_a = trm_b$ and $trm_a \preceq trm_b$ are wffs.
	\item if $trm_a$ and $trm_b$ are temporal terms, $trm_1, \dots, trm_n$ are non-temporal terms, and $r \in R$ are n-ary relation symbol,
	      then
	      \[
		      \text{TRUE}(trm_a, trm_b, r(trm_1, \dots, trm_n))
	      \]
	      is a wff.
	\item if $\phi_1$ and $\phi_2$ are wffs, then $\phi_1 \land \phi_2$ and $\neg \phi_1$ are wffs.
	\item if $\phi$ is a wff and $z \in TV \cup V$ is a variable, then $\forall z \phi$ is a wff.
\end{enumerate}

Again, we assume the usual definition of $\lor, \exists, \equiv, \supset$ and so on.

\begin{exmp} Here's an example sentences.
	\begin{equation}
		\text{TRUE}(t_1, t_2, \text{COLOR}(\text{HOUSE17}, \text{RED})))
	\end{equation}
	\begin{equation}
		\exists u \ \text{TRUE}(t_3, t_4, \text{ON}(u,B))
	\end{equation}
\end{exmp}

\subsubsection{Semantics}
An \textit{interpretation} is a tuple $\mathscr{S} = \langle  TW,\ \leqslant,\ W,\ TFN,\ FN,\ RL,\ M\rangle$ where $TW$
is a nonempty universe of time points, $\leqslant$ is a binary relation on $TW$, $W$ is a nonempty universe of individuals,
that is disjoint from $TW$, $TFN$ is a set of total functions $\bigcup_k (TW^k \rightarrow TW)$,
$FN$ is a set of total functions $ \bigcup_k (W^k \rightarrow W)$, $RL$ is a set of total relations over $W$, and
$M = \langle M1,\ M2,\ M3,\ M4,\ M5 \rangle$ is a meaning function as follows: $M_1 : T \rightarrow TW$,
$M_2: C \to W$, $M_3 : TF \to TFN$, $M_4 : TW \times TW \times F \to FN$, and $M_5: TW \times TW \times R \to RL$

A \textit{variable assignment} is a function $ VA = \langle VAT,\ VAV \rangle$, such that $VAT: VT \to TW$ and
$VAV: V \to W$. $M$ and $VA$ induce a time-dependent meaning $MVA$ on arbitrary terms in the following way.

We first define the meaning of arbitrary temporal terms. That meaning is the same regardless of when the terms are interpreted: the terms $1.1.2000$ and
$(12:00 + 12_{min})$ each denote a single, unambiguous absolute time. The precise
meaning of temporal terms is as follows. If $vt \in VT $, then $MVA (vt) = VAT (vt)$.
If $ct \in CT$, then $MVA (ct) = M_1(ct)$. If $f \in T F$ and $trm = f(trm_1, . . . , trm_n)$ Is a
temporal term, then

\[
	MVA(trm) = M_3(f)(MVA(trm_1), \dots, MVA(trm_n))
\]

\section{Galton and Allen's Temporal Logic}
In his paper \cite{galton2004}, Galton shows that Allen's Ontology is unsuitable for representing facts and continuous change.

\subsection{Properties and their negations}
In Allen's notation, the formula $\text{HOLDS}(p, T)$ says that the property $p$ holds over the interval $T$.
Allen introduces the axiom
\begin{equation}
	\text{HOLDS}(p, T) \leftrightarrow (\forall t) [\text{IN}(t, T) \to \text{HOLDS}(p, t)].
\end{equation}
Allen follows this with a more complicated axiom:

\begin{equation}
	\text{HOLDS}(p, T) \leftrightarrow (\forall t) [\text{IN}(t, T) \to (\exists s)[\text{IN}(t, T) \land \text{HOLDS}(p, s)]].
\end{equation}
And he defines the negation of a property as follows:
\begin{equation}
	\text{HOLDS}(\text{not}(p), T) \leftrightarrow \neg (\forall t) [\text{IN}(t, T) \to \neg \text{HOLDS}(p, t)].
\end{equation}
Allen notes that HOLDS(not($p$, $T$)) implies $\neg$ HOLDS($p$, $T$), but not conversely, and that HOLDS(not(not($p$)), $T$) is equivalent to HOLDS($p$, $T$).
