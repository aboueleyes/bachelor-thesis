\chapter{Log AS}

In his paper \cite{ismail2013stability}, Ismail presents
$Log_AS$ a class of many-sorted languages that share a common core of logical symbols and differ in a signature of non-logical symbols. In what follows, we identify a sort $\sigma$ with
the set of symbols of sort $\sigma$.

A $Log_AS$ language is a set of terms partitioned into three base syntactic types, $\sigma_S , \sigma_T ,$
and $\sigma_i.$ Intuitively, 
\begin{itemize}
  \item $\sigma_S$ is the set of terms denoting states 
   \item $\sigma_T$ is the set of terms denoting time points 
   \item $\sigma_I$ is the set of terms denoting anything else.
\end{itemize}

\section{Syntax}
An alphabet of $Log_A S$ is made up of a set of syncategorematic punctuation symbols and a set of denoting symbols each from a set $\sigma$ of syntactic types.

The set $\sigma$ is the smallest set containing all of the following types: 
$\sigma_S, \sigma_T, \sigma_I$ and $\varsigma_1 \to \varsigma_2$, 
for $\varsigma_1 \in \{\sigma_S, \sigma_T, \sigma_I \}$ and $\varsigma_2 \in \sigma$.
$\varsigma_1 \to \varsigma_2$ is the syntactic type of function symbols that take a single
argument of type $\sigma_S, \sigma_T$ or $\sigma_I$ and produce a functional
term of type $\varsigma_2$.
Given the restriction of the first argument of function symbols to base types, $Log_A S$ is a first-order language.


A $Log_A S$ alphabet is a union of four disjoint sets:
$ \Omega \cup \Xi \cup \Sigma \cup \Lambda$.

The set $\Omega$, the \textit{signature} of the language, is a nonempty set of constant and function symbols. Each symbol in the signature has a designated syntactic type from $\sigma$. $\Omega$  is
what distinguishes one $LogA S$ language from another.

The set $\Xi = \{x_i, t_i, s_i\}_{i \in \mathbb{N}}$ is a countably infinite set of variables, where$x_i \in \sigma_I, t_i \in \sigma_T$ and $s_i \in \sigma_S$. $\Sigma$ is a set
of syncategorematic symbols, including the comma, various
matching pairs of brackets and parentheses, and the symbol $\forall$.

The set $\Lambda$ is the set of logical symbols of $Log_A S$, defined
as the union of the following sets.

\begin{enumerate}
  \item $\neg \in \sigma_S \to \sigma_S$
  \item $\{\land, \lor \} \subseteq \sigma_S \to \sigma_S \to \sigma_S$
  \item $\textsc{HoldsAt} \in \sigma_S \to \sigma_T \to \sigma_S$
  \item $\prec \in \sigma_T \to \sigma_T \to \sigma_S$
\end{enumerate}

A $Log_AS$ language with signature $\Omega$ is denoted by $L_{\Omega}$ It
is the smallest set of terms formed according to the following
rules, where $\tau$ and $\tau_i (i \in \mathbb{N})$ are terms in $L_{\Omega}$

\begin{itemize}
  \item $\Xi \subset L_{\Omega}$
  \item $c \in L_{\Omega}$, where $c \in \Omega$ is a constant symbol.
  \item $f(\tau_1, \tau_2, \dots, \tau_n) \in L_{\Omega}$, where $f \in \Omega$ is of type 
    $\varsigma_1 \dots \to \varsigma_n \to \varsigma  (n > 0)$ and $\tau_i$ is of type of $\varsigma_i$.
  \item $\neg \tau \in L_{\Omega}$, where $\tau \in \sigma_S$.
  \item $(\tau_1 \otimes \tau_2) \in L_{\Omega}$, where $\otimes \in \{\land, \lor\}$ and $\tau_1, \tau_2 \in \sigma_S$.
  \item $\forall x(\tau) \in L_{\Omega}$, where $x \in \Xi$ and $\tau \in \sigma_S$.
  \item $\textsc{HoldsAt}(\tau_1, \tau_2) \in L_{\Omega}$, where $\tau_1 \in \sigma_S$ and $\tau_2 \in \sigma_T$.
  \item $\prec(\tau_1, \tau_2) \in L_{\Omega}$, where $\tau_1, \tau_2 \in \sigma_T$.
\end{itemize}




\section{Semantics}
