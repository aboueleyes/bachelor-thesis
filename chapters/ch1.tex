\chapter{Introduction}
The concept of time has long been a subject of philosophical inquiry, and computer science has increasingly shown interest in this area.
Philosophers and logicians have studied the nature of time, including questions of whether it is \textit{discrete} or \textit{continuous}, \textit{bounded} or \textit{unbounded}, and \textit{linear} or \textit{cyclical}.
Despite ongoing debates, people generally have a common-sense understanding of time, enabling them to navigate the world and communicate effectively.

Psychologists and philosophers distinguish between three categories of time: \textit{natural}, \textit{conventional}, and \textit{logical}.

\textit{Natural time} is based on the occurrence of natural phenomena, such as day and night and the seasons. In contrast, \textit{conventional time} is based on social conventions, such as the measurement of seconds, minutes, and hours.
\textit{Logical time} refers to the mathematical structure of time and its operations \cite{Galton1990-GALTLA-3}.

Temporal knowledge and reasoning are essential in many fields, including computer science, which is essential for designing information systems, verifying programs, and developing artificial intelligence applications.

In discussions of aspectual phenomena, \textit{events} are another primary type of entity that can be attributed to temporal incidence.
Unlike \textit{states}, events \cite{vendler1957verbs} occur and are bounded in time, having a distinct beginning and end. Events are a fundamental concept in many areas of artificial intelligence and related fields, such as situation calculus and event calculus.
Events are also recognized in other domains, such as linguistics, where they are seen as the basic building blocks of narrative discourse.

Therefore, an investigation of events is crucial in understanding various cognitive and linguistic phenomena, such as the representation of causation, the interpretation of tense and aspect, and the construction of narratives.


In this thesis, I introduce $Log_A$T, a family of languages designed for the purpose of reasoning about temporal ontologies, encompassing states and events.
$Log_A$T is algebraic in the sense that it relies solely on terms rather than sentences.

$Log_A$T is an extension of $Log_A$S, which is a temporal language that can be used to represent and reason about temporal ontologies.
$Log_A$S is based on the notion of \textit{states}.
$Log_A$S and $Log_A$T are variants of $Log_A$B \cite{log_ab}, which is an algebraic language
for reasoning about propositions and propositional attitudes.

% \section{Thesis Chapters}
% TO BE FILLED IN.
