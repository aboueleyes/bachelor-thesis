\chapter{States and Events}
\section{Verbs and Time}
\subsection{Introduction}

In his paper "Verbs and Time" \cite{vendler1957verbs}, Vendler proposes a categorization of verbs based on their temporal and aspectual properties. The goal of this categorization is to provide a framework for understanding how verbs relate to time and how their meanings are structured.

\subsection{Background}

Verbs are a central component of natural language and are used to express actions, states, and events. They are also crucial for representing the temporal structure of sentences, as they allow us to talk about past, present, and future events. However, not all verbs are the same when it comes to their temporal properties. Some verbs, like "run" or "sing", describe actions that occur over a period of time, while others, like "know" or "believe", describe states or attitudes that hold at a particular point in time.

\subsection{Vendler's Categorization}

Vendler proposes a categorization of verbs based on their temporal and aspectual properties. He identifies four main categories of verbs: \textit{activities}, \textit{accomplishments}, \textit{achievements}, and \textit{states}.

\textbf{Activities} are verbs that describe ongoing processes or actions that have no inherent endpoint. Examples of activities include "run", "swim", and "write".


\textbf{Accomplishments} are verbs that describe actions that have a clear endpoint or goal. Examples of accomplishments include "build", "solve", and "paint".


\textbf{Achievements} are verbs that describe events that happen suddenly or instantaneously, without any inherent duration. Examples of achievements include "win", "die", and "arrive".


\textbf{States} are verbs that describe conditions or states of being that hold over a period of time. Examples of states include "know", "believe", and "love".

Each of these categories has a different temporal and aspectual structure, and Vendler argues that these differences are crucial for understanding the meaning of verbs and their relationships to time.



\begin{exmp} Analysis of the verb "think"
\begin{enumerate}
  \item He is "thinking" about Ibrahim. \label{first}
  \item He "thinks" that Ibrahim is a philosopher. \label{second}
\end{enumerate}

The verb "think" functions differently in the two sentences presented. In the first sentence \ref{first}, "thinking" is characterized as a \textit{process} that involves ongoing mental activity related to Ibrahim, whereas in the second sentence \ref{second}, "thinks" describes a \textit{state} of belief or opinion about Ibrahim.

 The first sentence \ref{first} suggests that the activity of "thinking" about something is a deliberate and continuous process that occurs over time. The phrase "deliberately" highlights the intentional nature of the activity. Furthermore, if one were to think about Ibrahim for a period of time, it would imply that the thinking was consistent and continuous throughout that time period. This notion is expressed as \textit{homogeneity}.
\end{exmp}


\section{Interval Algebra}

Allen's paper presents a framework called the \textit{Interval Algebra} for representing and reasoning about actions and time in natural language processing and artificial intelligence systems \cite{allen1984towards}. The Interval Algebra uses a set of basic temporal relations to describe the temporal and causal relationships between events and actions. For example, it can represent relations such as before,'' after,'' meets,'' overlaps,'' during,'' and equals'' between different intervals of time.

Allen argues that the Interval Algebra can be used to model a wide range of natural language expressions involving \textit{actions} and \textit{events}. This includes simple statements about temporal order, such as Ibrahim ate breakfast before he went to work,'' as well as more complex descriptions of causality and concurrency, such as When Ibrahim opened the door, the alarm went off.''


\pagebreak
The logic used is a typed first-order logic predicate calculus.

\begin{exmp} Allen's Interval Algebra

\begin{enumerate}
  \item DURING($t_1$, $t_2$) time interval $t_1$ is fully contained within time interval $t_2$.

  \item STARTS($t_1$, $t_2$) time interval $t_1$ shares a starting point with time interval $t_2$ but ends before $t_2$.

  \item FINISHES($t_1$, $t_2$) time interval $t_1$ shares the same end as time interval $t_2$ but starts after $t_2$.

  \item IN($t_1$, $t_2$) $\iff$ (DURING($t_1$, $t_2$) $\lor$ STARTS($t_1$, $t_2$) $\lor$ FINISHES($t_1$, $t_2$)). 

  \item HOLDS(\(p\), \(t\)) is true iff \(p\) holds during \(t\).
\end{enumerate}



\begin{center}
  HOLDS(\(p\), \(T\)) \(\iff ( \forall t \) IN $ (t,T) \implies $ HOLDS($ p, T $)).
\end{center}

This definition states that HOLDS($p$, $T$) is true iff $p$ holds during every subinterval $t$ of $T$. In other words, $p$ holds throughout the entire interval $T$.
\end{exmp}

\subsection{Occurrences}
Allen divides the class of \textit{occurrences} into two subclasses: \textit{events} and \textit{processes}.
\begin{enumerate}
  \item \textbf{Events} describe an activity that involves an outcome. Examples of events such as "Ibrahim walked to the store".

  \item \textbf{Processes} refer to some activity not associated with a result. Examples of processes include "Ibrahim was walking" and "Ibrahim was running".
\end{enumerate}
One way to distinguish between \textit{events} and \textit{processes} is that one can count how many times an event occurs, but one cannot count 
the number of times a process occurs.

Another way is to consider the characteristics of the set of temporal intervals that they hold or occur over. 
consider the process "I am walking" over the Interval $I$. Unlike events, this process might be occurring over subintervals of $I$.

The predicate OCCUR takes an event and a time interval and is true if the event happens over the time interval $t$ and there is no subinterval of $t$
over which the event occurs.

\begin{center}
  OCCUR($e$, $t$) $\land$ IN($t^\prime$, $t$) $\implies$ \(\lnot\) OCCUR($e$, $t^\prime$).
\end{center}

