\chapter{States and Events}
\section{Verbs and Time}
\subsection{Introduction}

In his paper "Verbs and Times" \cite{vendler1957verbs}, Vendler proposes a categorization of verbs based on their temporal and aspectual properties. The goal of this categorization is to provide a framework for understanding how verbs relate to time and how their meanings are structured.

\subsection{Background}

Verbs are a central component of natural language and are used to express actions, states, and events. They are also crucial for representing the temporal structure of sentences, as they allow us to talk about past, present, and future events. However, not all verbs are the same when it comes to their temporal properties. Some verbs, like "run" or "sing", describe actions that occur over a period of time, while others, like "know" or "believe", describe states or attitudes that hold at a particular point in time.

\subsection{Vendler's Categorization}

Vendler proposes a categorization of verbs based on their temporal and aspectual properties. He identifies four main categories of verbs: \textit{activities}, \textit{accomplishments}, \textit{achievements}, and \textit{states}.

\textbf{Activities} are verbs that describe ongoing processes or actions that have no inherent endpoint. Examples of activities include "run", "swim", and "write".


\textbf{Accomplishments} are verbs that describe actions that have a clear endpoint or goal. Examples of accomplishments include "build", "solve", and "paint".


\textbf{Achievements} are verbs that describe events that happen suddenly or instantaneously, without any inherent duration. Examples of achievements include "win", "die", and "arrive".


\textbf{States} are verbs that describe conditions or states of being that hold over a period of time. Examples of states include "know", "believe", and "love".

Each of these categories has a different temporal and aspectual structure, and Vendler argues that these differences are crucial for understanding the meaning of verbs and their relationships to time.



\begin{exmp} Analysis of the verb "think"
\begin{enumerate}
  \item He is "Thinking" about Ibrahim. \label{first}
  \item He "thinks" that Ibrahim is a philosopher. \label{second}
\end{enumerate}

The verb "think" functions differently in the two sentences presented. In the first sentence \ref{first}, "thinking" is characterized as a \textit{process} that involves ongoing mental activity related to Ibrahim, whereas in the second sentence \ref{second}, "thinks" describes a \textit{state} of belief or opinion about Ibrahim.

 The first sentence \ref{first} describes an \textit{activity} or process that is ongoing and intentional, while the second sentence \ref{second} does not involve any active engagement. This illustrates the distinction between the two different senses of the verb "think."

 The first sentence \ref{first} suggests that the activity of "thinking" about something is a deliberate and continuous process that occurs over time. The phrase "deliberately" highlights the intentional nature of the activity. Furthermore, if one were to think about Ibrahim for a period of time, it would imply that the thinking was consistent and continuous throughout that time period. This notion is expressed as \textit{homogeneity}.
\end{exmp}



