\chapter{Events}

Up until now, we have only considered \textit{states} in our representation using $Log_A S$. However, states alone cannot represent certain situations, such as the distinction between the following two sentences:

\begin{enumerate}[label=(\arabic*)]
    \item I ate an apple.
    \item I was eating an apple.
\end{enumerate}

\begin{equation}
    \textsc{HoldsAt}(\textsc{Ate}(I, \textsc{Apple}), t_1, t_2)
    \label{eq:ate}
\end{equation}
Sentence (1) uses the perfective aspect and presents the situation as a complete, finished event, while sentence (2) uses the imperfective aspect and focuses on the ongoing nature of the situation without indicating whether it has ended.

To represent events, we need to extend our language. Equation~\ref{eq:ate} represents a state, but we need to represent an event, so we need to introduce events into our language
which we will call $Log_A T$.

\section{Tokens and Categories of Events}
\subsection{Event Tokens}

In contrast to states, event tokens do not hold homogeneously over time.
Instead, they occur as complete wholes.
To capture this notion, we introduce a new sort $\mathcal{ET}$ for event tokens and a new function $\textsc{Occurs}$ that relates events to the time they occur.

\begin{defn}
    The function $\textsc{Occurs}: \mathcal{ET} \times \mathcal{T} \times \mathcal{T} \to \textsc{Eter}$ maps an event token,
    a starting time, and an ending time to an eternal state.
\end{defn}

\begin{defn}
    \textlbrackdbl $\textsc{Occurs}(\tau_1, \tau_2, \tau_3)$\textrbrackdbl$^{\mathcal{V}}$ = $\mathfrak{o}$(\textlbrackdbl $\tau_1$\textrbrackdbl$^{\mathcal{V}}$, \textlbrackdbl $\tau_2$\textrbrackdbl$^{\mathcal{V}}$, \textlbrackdbl$\tau_3$\textrbrackdbl$^{\mathcal{V}}$)
\end{defn}

The occurrence of an event is eternal, meaning that if an event occurred between two time points, it will always have occurred between those two time points. Additionally, each occurrence is unique, meaning that an event token occurs only once. This is captured by the following axiom:

To ensure that an event token occurs only once between any two time points, we add the following axiom.

\begin{axiom}
    \begin{equation}
        \mathfrak{o}(e, t_1, t_2) \cdot \mathfrak{o}(e, t_3, t_4) = \bot  \ if \ t_1 \neq t_3 \ or \  t_2 \neq t_4
    \end{equation}
    \label{ax:occurs_once}
\end{axiom}

Intuitively, $\mathfrak{o}(e_1, t_1, t_2)$ means that the event token $e$ occurs at some time between $t_1$ and $t_2$. We also add an axiom to ensure that an event token occurring between $t_1$ and $t_2$ is followed by $t_2$:

\begin{axiom}
    \begin{equation}
        \mathfrak{o}(e, t_1, t_2) \cdot (t_2 < t_1) = \bot
    \end{equation}
\end{axiom}

\begin{theorem} Events are heterogeneous.
    \begin{equation}
        \mathfrak{o}(e, t_1, t_2) \leq \prod_{t_3, t_4 \in \mathcal{T}} [(t_3 < t_2) \cdot (t_4 < t_1)
            \leq - \mathfrak{o}(e, t_3, t_4)].
        \label{eq:events_heterogeneous}
    \end{equation}
\end{theorem}
\begin{theorem}
    \begin{equation}
        \mathfrak{o}(e, t_1, t_2) \leq \prod_{t_3, t_4 \in \mathcal{T}} [(t_3 < t_1) \cdot (t_2 < t_4)
            \leq - \mathfrak{o}(e, t_3, t_4)].
        \label{eq:events_heterogeneous2}
    \end{equation}
\end{theorem}

Both Equation~\ref{eq:events_heterogeneous} and Equation~\ref{eq:events_heterogeneous2} follows from the
axiom~\ref{ax:occurs_once}.

That means that if an event occurs over some interval, it cannot occur over any super or sub-interval.
We think of the interval as the smallest part the event can fit in.\footnote{Allen \cite{allen1984towards} defines events in a similar way, if an event occurs over some interval, it cannot occur over any sub-interval.
}

\subsection{Event Categories}
We can easily recognize that some event token is \textit{eating an apple}. So we will add this notion of \textit{category}\footnote{VEL has a similar notion called \textit{event radicals} \cite{bennett2001unifying}} to our language.

Let $\mathcal{EC}$ be the set of event categories. We will use the following function to relate event tokens to event categories.

\[
    \textsc{Cat}: \mathcal{ET} \times \mathcal{EC} \to \mathcal{S}
\]

\begin{defn}
    \textlbrackdbl $\textsc{Cat}(\tau_1, \tau_2)$\textrbrackdbl$^{\mathcal{V}}$ = $\mathfrak{c}$(\textlbrackdbl $\tau_1$\textrbrackdbl$^{\mathcal{V}}$, \textlbrackdbl $\tau_2$\textrbrackdbl$^{\mathcal{V}}$)
\end{defn}

Unlike $\textsc{Occurs}$, $\textsc{Cat}$ does not always return eternal states.

//TODO: add justification for this.
%%TODO: add justification for this.

An event token can belong to multiple categories, thus $\textsc{Cat}$ is a many-to-many relation,a as for example an event of
\textit{driving} can be an event of \textit{commuting} at the same time.

\section{Punctual and Durative Events}

The sort $\mathcal{ET}$ for event tokens is partitioned into two subsets: $\mathcal{ET}_d$ for durative events and $\mathcal{ET}_p$ for punctual events.
Similarly, the set of event categories $\mathcal{EC}$ is partitioned into $\mathcal{EC}_d$ and $\mathcal{EC}_p$.

Punctual events is the \textit{onset} or \textit{cessation} of some state. On the other hand, a durative event is made up
of the punctual event event of some state starting to hold, the state's holding for some time, and the punctual event of the state ceasing.

A state may be associated with event categories $\uparrow s$ and $\downarrow s$ for the onset and cessation of the state, respectively.

\begin{defn}
    \begin{equation}
        \textsc{Onset} : \textsc{Perm} \cup \textsc{Temp} \to^{into} \mathcal{EC}_p
    \end{equation}
\end{defn}
\begin{defn}
    \begin{equation}
        \textsc{Cease} : \textsc{Co-Perm} \cup \textsc{Temp} \to^{into} \mathcal{EC}_p
    \end{equation}
\end{defn}

\begin{defn}
    \[
        \textlbrackdbl \textsc{Onset}(\tau)\textrbrackdbl^{\mathcal{V}}
        = \uparrow(\textlbrackdbl \tau \textrbrackdbl^{\mathcal{V}})
    \]
\end{defn}
\begin{defn}
    \[
        \textlbrackdbl \textsc{Cease}(\tau)\textrbrackdbl^{\mathcal{V}}
        = \downarrow(\textlbrackdbl \tau \textrbrackdbl^{\mathcal{V}})
    \]

\end{defn}

Intuitively, $\uparrow s$ is equivalent to $\downarrow(-s)$, and $\downarrow s$ is equivalent to $\uparrow(-s)$, Let's add the following axioms to our theory.
\begin{axiom}
    \begin{equation}
        \mathfrak{c}(e_p, \uparrow s) \cdot (\mathfrak{c}(e_p, \downarrow(- s))) = \top.
    \end{equation}
\end{axiom}
\begin{axiom}\label{ax:cessation}
    \begin{equation}
        \mathfrak{c}(e_p, \downarrow s) \cdot (\mathfrak{c}(e_p, \uparrow(- s))) = \top.
    \end{equation}
\end{axiom}

Using the above axioms, we can prove the following theorem which suggests that the arrow flip with each negation.

\begin{theorem}\label{thm:arrow_flip}
    \begin{equation}
        \mathfrak{c}(e_p, \uparrow s) \cdot \mathfrak{c}(e_p, \downarrow (-- s)) = \top.
    \end{equation}
\end{theorem}
\begin{theorem}
    \begin{equation}
        \mathfrak{c}(e_p, \downarrow s) \cdot \mathfrak{c}(e_p, \uparrow (-- s)) = \top.
    \end{equation}
\end{theorem}

Both axioms and theorems above are not enough to tell how the occurrence of onsets and cessations are related to states.
Intuitively, any
transition is immediately preceded by a state and immediately followed by its complement (or vice versa).

\begin{axiom}\label{ax:transition}
    \begin{equation}
        \mathfrak{o}(e_p, t_1, t_2) \cdot \mathfrak{c}(e_p, \uparrow s) \leq \sum_{t_3, t_4 \in \mathcal{T}}[(t_2 < t_3) \cdot \mathfrak{h}(s, t_3, t_4)]
    \end{equation}
\end{axiom}
\begin{axiom}
    \begin{equation}
        \mathfrak{o}(e_p, t_1, t_2) \cdot \mathfrak{c}(e_p, \uparrow s) \leq \sum_{t_3, t_4 \in \mathcal{T}} [(t_4 < t_1) \cdot \mathfrak{h}(-s, t_3, t_4)]
    \end{equation}
\end{axiom}

The above axioms was for the onset of a state. We can similarly prove theorems for the cessation of a state.

\begin{theorem}\label{thm:cessation}
    \begin{equation}
        \mathfrak{o}(e_p, t_1, t_2) \cdot \mathfrak{c}(e_p, \downarrow s) \leq \sum_{t_3, t_4 \in \mathcal{T}}[(t_2 < t_3) \cdot \mathfrak{h}(s, t_3, t_4)]
    \end{equation}
\end{theorem}
\begin{theorem}\label{thm:cessation_2}
    \begin{equation}
        \mathfrak{o}(e_p, t_1, t_2) \cdot \mathfrak{c}(e_p, \downarrow s) \leq \sum_{t_3, t_4 \in \mathcal{T}} [(t_4 < t_1) \cdot \mathfrak{h}(-s, t_3, t_4)]
    \end{equation}
\end{theorem}

\begin{proof}
    \begin{align*}
        \mathfrak{o}(e_p, t_1, t_2) \cdot \mathfrak{c}(e_p, \downarrow s)                                                                                     \\
         & = \mathfrak{o}(e_p, t_1, t_2) \cdot \mathfrak{c}(e_p, \uparrow (-s))  \hspace{0.29\linewidth  }  \text{Axiom~\ref{ax:cessation}}                   \\
         & \leq \sum_{t_3, t_4 \in \mathcal{T}}[(t_2 < t_3) \cdot \mathfrak{h}(-(-s), t_3, t_4)] \hspace{0.204 \linewidth}   \text{Axiom~\ref{ax:transition}} \\
         & \leq \sum_{t_3, t_4 \in \mathcal{T}}[(t_2 < t_3) \cdot \mathfrak{h}(s, t_3, t_4)]. \hspace{0.255 \linewidth}   \text{Theorem~\ref{thm:arrow_flip}}
    \end{align*}
\end{proof}

A sufficient condition for the occurrence of onsets is that a state does not hold followed by its complement. We can capture this condition by the following axiom.

\begin{axiom}\label{ax:transition_2}
    \begin{equation}
        - \mathfrak{h}(s, t_1, t_2) \cdot \mathfrak{h}(s, t_3, t_4) \cdot (t_2 < t_3)\leq \mathfrak{o}(e_p, t_2, t_3) \cdot \mathfrak{c}(e_p, \uparrow s).
    \end{equation}
\end{axiom}

We can prove a similar theorem for the cessation of a state.

\begin{theorem}\label{thm:cessation_3}
    \begin{equation}
        \mathfrak{h}(s, t_1, t_2) \cdot - \mathfrak{h}(s, t_3, t_4) \cdot (t_2 < t_3)\leq \mathfrak{o}(e_p, t_2, t_3) \cdot \mathfrak{c}(e_p, \downarrow s).
    \end{equation}
    
\end{theorem}

\begin{proof} 
    \begin{align*}
        \mathfrak{h}(s, t_1, t_2) \cdot - \mathfrak{h}(s, t_3, t_4) \cdot (t_2 < t_3) \\
        &= \mathfrak{h}(-- s, t_1, t_2) \cdot - \mathfrak{h}(s, t_3, t_4) \cdot (t_2 < t_3) \\
        &= - \mathfrak{h}(- s, t_1, t_2) \cdot - \mathfrak{h}(s, t_3, t_4) \cdot (t_2 < t_3) \\
        &= - \mathfrak{h}(s, t_1, t_2)   \cdot \mathfrak{h}(-s, t_3, t_4) \cdot (t_2 < t_3) \\
        & \leq \mathfrak{o}(e_p, t_2, t_3) \cdot \mathfrak{c}(e_p, \uparrow - s) \hspace{0.06 \linewidth} \text{Axiom~\ref{ax:transition_2}} \\
        &\leq \mathfrak{o}(e_p, t_2, t_3) \cdot \mathfrak{c}(e_p, \downarrow s) \hspace{0.06 \linewidth}  \ \ \text{Theorem~\ref{thm:arrow_flip}}
    \end{align*}
\end{proof}

We can add more restrictions based on the type of state. For example, if you are considering
a $\textsc{Prem}$ state, we can prove once an onset happens, no other onset can happen.

\begin{theorem}
    \begin{equation}
        \mathfrak{o}(e_{p1}, t_1, t_2) \cdot \mathfrak{c}(e_p, \uparrow s^{p}) \leq 
        \prod_{\substack{e_{p2} \in \mathcal{E}_p \\ t_3, t_4 \in \mathcal{T}}}[(t_2 < t_3) \cdot \mathfrak{o}(e_{p2}, t_3, t_4) ]
        \leq - \mathfrak{c}(e_{p2}, \uparrow s^{p})
    \end{equation}
\end{theorem}

Similarly, we can do the mirror for $\textsc{Co-Perm}$ states.

\begin{theorem}
    \begin{equation}
        \mathfrak{o}(e_{p1}, t_1, t_2) \cdot \mathfrak{c}(e_p, \downarrow s^{cp}) \leq 
        \prod_{\substack{e_{p2} \in \mathcal{E}_p \\ t_3, t_4 \in \mathcal{T}}}[(t_2 < t_3) \cdot \mathfrak{o}(e_{p2}, t_3, t_4) ]
        \leq - \mathfrak{c}(e_{p2}, \downarrow s^{cp})
    \end{equation}
\end{theorem}