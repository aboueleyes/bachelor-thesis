\chapter{Events}

We have only considered \textit{states} in our representation using $Log_A$S. However, states alone cannot represent certain situations, such as the distinction between the following two sentences:

\begin{enumerate}[label=(\arabic*)]
	\item I ate an apple.
	\item I was eating an apple.
\end{enumerate}

\begin{equation}
	\textsc{HoldsAt}(\textsc{Ate}(I, \textsc{Apple}), t_1, t_2)
	\label{eq:ate}
\end{equation}
Sentence (1) uses the perfective aspect and presents the situation as a whole, finished event, while sentence (2) uses the imperfective aspect and focuses on the ongoing nature of the situation without indicating whether it has ended.

To represent events, we need to extend our language. Sentence~\ref{eq:ate} represents a state, but we need to represent an event, so we need to introduce events into our language
which we will call $Log_A$T.

\section{Tokens and Categories of Events}
\subsection{Event Tokens}

In contrast to states, event tokens do not hold homogeneously over time.
Instead, they occur as complete wholes.
To capture this notion, we introduce a new sort $\sigma_\mathcal{ET}$ for event tokens and a new function term $\textsc{Occurs}$ that relates events to the time they occur.

The function symbol $\textsc{Occurs}: \sigma_\mathcal{ET} \to \sigma_\mathcal{T} \to \sigma_\mathcal{T} \to \sigma_\mathcal{S}$ maps an event token,
a starting time, and an ending time to an eternal state.

$$\mathfrak{o} : \mathcal{ET} \times \mathcal{T} \times \mathcal{T} \to \textsc{Eter}$$ where
\begin{itemize}
	\item
	      Each occurrence is unique, meaning that an event token occurs only once.
	      \begin{equation}\label{ax:occurs_once}
		      \prod\footnote{
			      Here I use this notation of having multiple quantifiers in a single axiom to make the axiom more readable.
			      But this shorthand notation is equivalent to
			      $ \displaystyle \prod_{e \in \mathcal{ET}} \prod_{t_1 \in \mathcal{T}}
				      \prod_{t_2 \in \mathcal{T}}\prod_{t_2 \in \mathcal{T}}\prod_{t_3 \in \mathcal{T}}$}_{\substack{e \in \mathcal{ET} \\ t_1, t_2, t_3, t_4 \in \mathcal{T}}} \mathfrak{o}(e, t_1, t_2) \cdot \mathfrak{o}(e, t_3, t_4) = \bot  \ if \ t_1 \neq t_3 \ or \  t_2 \neq t_4
	      \end{equation}
	\item
	      Intuitively, $\mathfrak{o}(e_1, t_1, t_2)$ means that the event token $e$ occurs at some time between $t_1$ and $t_2$.
	      We also add an axiom to ensure that an event token occur between $t_1$ and $t_2$ if $t_1$ is followed $t_2$.
	      \begin{equation}
		      \prod_{\substack{e_d \in \mathcal{ET}_d \\ t_1, t_2 \in \mathcal{T}}} \mathfrak{o}(e_d, t_1, t_2) \cdot (t_2 < t_1) = \bot
	      \end{equation}
	      \begin{equation}
		      \prod_{\substack{e_p \in \mathcal{ET}_p \\ t_1, t_2 \in \mathcal{T}}} \mathfrak{o}(e_p, t_1, t_2) \cdot (t_2 = t_1) = \bot \quad \text{for every} \quad t_1 \neq t_2
	      \end{equation}
\end{itemize}

\textlbrackdbl $\textsc{Occurs}(\tau_1, \tau_2, \tau_3)$\textrbrackdbl$^{\mathcal{V}}$ = $\mathfrak{o}$(\textlbrackdbl $\tau_1$\textrbrackdbl$^{\mathcal{V}}$, \textlbrackdbl $\tau_2$\textrbrackdbl$^{\mathcal{V}}$, \textlbrackdbl$\tau_3$\textrbrackdbl$^{\mathcal{V}}$)

The occurrence of an event is eternal, meaning that if an event occurred between two time points, it would always have occurred between those two time points.


% \begin{axiom}\label{ax:occurs_once}
%     \begin{equation}
%         \prod_{\substack{e \in \mathcal{ET} \\ t_1, t_2, t_3, t_4 \in \mathcal{T}}} \mathfrak{o}(e, t_1, t_2) \cdot \mathfrak{o}(e, t_3, t_4) = \bot  \ if \ t_1 \neq t_3 \ or \  t_2 \neq t_4
%     \end{equation}
% \end{axiom}


% \begin{axiom}
%     \begin{equation}
%         \prod_{\substack{e_d \in \mathcal{ET}_d \\ t_1, t_2 \in \mathcal{T}}} \mathfrak{o}(e_d, t_1, t_2) \cdot (t_2 < t_1) = \bot
%     \end{equation}
%     \begin{equation}
%         \prod_{\substack{e_p \in \mathcal{ET}_p \\ t_1, t_2 \in \mathcal{T}}} \mathfrak{o}(e_p, t_1, t_2) \leq t_2 = t_1.
%     \end{equation}
% \end{axiom}

\begin{theorem}\label{thm:events_heterogeneous} Events are heterogeneous.
	\begin{equation}
		\prod_{\substack{e \in \mathcal{ET} \\ t_1, t_2 \in \mathcal{T}}} [\mathfrak{o}(e, t_1, t_2) \cdot \prod_{t_3, t_4 \in \mathcal{T}} [(t_1 < t_3) \cdot (t_4 < t_2)]]
		\leq - \mathfrak{o}(e, t_3, t_4).
		\label{eq:events_heterogeneous}
	\end{equation}
\end{theorem}
\begin{theorem}\label{thm:events_heterogeneous2}
	\begin{equation}
		\prod_{\substack{e \in \mathcal{ET} \\ t_1, t_2 \in \mathcal{T}}}[ \mathfrak{o}(e, t_1, t_2) \cdot \prod_{t_3, t_4 \in \mathcal{T}} [(t_3 < t_1) \cdot (t_2 < t_4)]]
		\leq - \mathfrak{o}(e, t_3, t_4).
		\label{eq:events_heterogeneous2}
	\end{equation}
\end{theorem}

Both Theorem~\ref{thm:events_heterogeneous} and Theorem~\ref{thm:events_heterogeneous2} follow from~\ref{ax:occurs_once}.

That means an event cannot occur over any super or sub-interval if it occurs over some interval.
We think of the interval as the smallest part the event can fit in.\footnote{Allen \cite{allen1984towards} defines events similarly; if an event occurs over some interval, it cannot occur over any sub-interval.
}

\subsection{Event Categories}
We can easily recognize that some event token is \textit{eating an apple}. So we will add this notion of \textit{category}\footnote{VEL has a similar notion called \textit{event radicals} \cite{bennett2001unifying}} to our language.

Let $\sigma_\mathcal{EC}$ be the sort of event categories. We will use the following function term to relate event tokens to event categories.

\[
	\textsc{Cat}: \sigma_\mathcal{ET} \to \sigma_\mathcal{EC} \to \sigma_\mathcal{S}
\]
$$\mathfrak{c} : \mathcal{ET}_p \times \mathcal{EC}_p \to \textsc{Temp}$$
$$\mathfrak{c} : \mathcal{ET}_d \times \mathcal{EC}_d \to \textsc{Temp}$$

\textlbrackdbl $\textsc{Cat}(\tau_1, \tau_2)$\textrbrackdbl$^{\mathcal{V}}$ = $\mathfrak{c}$(\textlbrackdbl $\tau_1$\textrbrackdbl$^{\mathcal{V}}$, \textlbrackdbl $\tau_2$\textrbrackdbl$^{\mathcal{V}}$)

Unlike $\textsc{Occurs}$, $\textsc{Cat}$ does not always return eternal states.\footnote{If John shot Adam, and Adam died after 3 days, then that event token will only be of category \textit{killing} after 3 days \cite{timeofakilling}.}

An event token can belong to multiple categories; thus $\textsc{Cat}$ is a many-to-many relation, as for example, an event of
\textit{driving} can be an event of \textit{commuting} at the same time.

\section{Punctual and Durative Events}

The set $\mathcal{ET}$ for event tokens is partitioned into two subsets: $\mathcal{ET}_d$ for durative events and $\mathcal{ET}_p$ for punctual events.
Similarly, the set of event categories $\mathcal{EC}$ is partitioned into $\mathcal{EC}_d$ and $\mathcal{EC}_p$.

Punctual events are the \textit{onsets} or \textit{cessations} of states. On the other hand, a durative event is made up
of the punctual event of some state starting to hold, the state's holding for some time, and the punctual event of the state ceasing.

A state may be associated with event categories $\uparrow s$ and $\downarrow s$ for the onset and cessation of the state, respectively.

\begin{equation}
	\textsc{Onset} : \sigma_{\mathcal{S}} \to \sigma_{\mathcal{EC}}
\end{equation}
\begin{equation}
	\textsc{Cease} : \sigma_{\mathcal{S}} \to \sigma_{\mathcal{EC}}
\end{equation}

$$\uparrow :  \textsc{Temp} \cup \textsc{Perm} \to^{into} \mathcal{EC}_p$$
$$\downarrow :  \textsc{Temp} \cup \textsc{Co-Perm} \to^{into} \mathcal{EC}_p$$
Intuitively, $\uparrow s$ is equivalent to $\downarrow(-s)$, and $\downarrow s$ is equivalent to $\uparrow(-s)$. Let us add these restrictions on the semantics.
\begin{equation}
	\prod_{\substack{e_p \in \mathcal{ET}_p \\ s \in \mathcal{S}}} \mathfrak{c}(e_p, \uparrow s) = (\mathfrak{c}(e_p, \downarrow(- s))).
\end{equation}
\begin{equation}\label{ax:cessation}
	\prod_{\substack{e_p \in \mathcal{ET}_p \\ s \in \mathcal{S}}}\mathfrak{c}(e_p, \downarrow s) = (\mathfrak{c}(e_p, \uparrow(- s))).
\end{equation}

\[
	\textlbrackdbl\textlbrackdbl \textsc{Onset}(\tau) \textrbrackdbl \textrbrackdbl^{\mathcal{V}}
	= \uparrow(\textlbrackdbl \textlbrackdbl\tau \textrbrackdbl\textrbrackdbl^{\mathcal{V}})
\]

\[
	\textlbrackdbl\textlbrackdbl \textsc{Cease}(\tau) \textrbrackdbl\textrbrackdbl^{\mathcal{V}}
	= \downarrow(\textlbrackdbl \textlbrackdbl\tau \textrbrackdbl\textrbrackdbl^{\mathcal{V}})
\]


Using the above semantics, we can prove the following theorem which suggests that the arrow flips with each negation.

\begin{theorem}\label{thm:arrow_flip}
	\begin{equation}
		\prod_{\substack{e_p \in \mathcal{ET}_p \\ s \in \mathcal{S}}} \mathfrak{c}(e_p, \uparrow s) = \mathfrak{c}(e_p, \downarrow (-- s)).
	\end{equation}
\end{theorem}
\begin{theorem}
	\begin{equation}
		\prod_{\substack{e_p \in \mathcal{ET}_p \\ s \in \mathcal{S}}}\mathfrak{c}(e_p, \downarrow s) = \mathfrak{c}(e_p, \uparrow (-- s)).
	\end{equation}
\end{theorem}

Both theorems above are not enough to tell how the occurrence of onsets and cessations are related to states.
Intuitively, any
transition is immediately preceded by a state and immediately followed by its complement (or vice versa).

\begin{equation}\label{ax:transition}
	\prod_{\substack{e_p \in \mathcal{ET}_p \\ s \in \mathcal{S} \\ t_1 \in \mathcal{T}}} \mathfrak{o}(e_p, t_1, t_1) \cdot \mathfrak{c}(e_p, \uparrow s) \cdot
	\sum_{t_3 \in \mathcal{T}}\prod_{t_2 \in \mathcal{T}} (t_1 < t_2)  \leq \mathfrak{h}(s, t_2, t_3)
\end{equation}
\begin{equation}
	\prod_{\substack{e_p \in \mathcal{ET}_p \\ s \in \mathcal{S} \\ t_1 \in \mathcal{T}}}\mathfrak{o}(e_p, t_1, t_1) \cdot \mathfrak{c}(e_p, \uparrow s) \cdot
	\sum_{t_3 \in \mathcal{T}} \prod_{t_2 \in \mathcal{T}} (t_2 < t_1)  \leq \mathfrak{h}(-s, t_3, t_2)
\end{equation}

The above restrictions were for the onset of a state. We can similarly prove theorems for the cessation of a state.

\begin{theorem}\label{thm:cessation}
	\begin{equation}
		\prod_{\substack{e_p \in \mathcal{ET}_p \\ s \in \mathcal{S} \\ t_1 \in \mathcal{T}}}\mathfrak{o}(e_p, t_1, t_1) \cdot \mathfrak{c}(e_p, \downarrow s) \cdot
		\sum_{t_3 \in \mathcal{T}} \prod_{t_2 \in \mathcal{T}} (t_2 < t_1) \leq \mathfrak{h}(s, t_3, t_2)
	\end{equation}
\end{theorem}
\begin{theorem}\label{thm:cessation_2}
	\begin{equation}
		\prod_{\substack{e_p \in \mathcal{ET}_p \\ s \in \mathcal{S} \\ t_1 \in \mathcal{T}}}  \mathfrak{o}(e_p, t_1, t_1) \cdot \mathfrak{c}(e_p, \downarrow s) \cdot
		\sum_{t_3 \in \mathcal{T}} \prod_{t_2 \in \mathcal{T}} (t_1 < t_2) \leq \mathfrak{h}(-s, t_2, t_3)
	\end{equation}
\end{theorem}

\begin{proof}
	\begin{align*}
		\prod_{\substack{e_p \in \mathcal{ET}_p                                                               \\ s \in \mathcal{S} \\ t_1 \in \mathcal{T}}}\mathfrak{o}(e_p, t_1, t_1) \cdot \mathfrak{c}(e_p, \downarrow s)\\    \cdot
		\sum_{t_3 \in \mathcal{T}} \prod_{t_2 \in \mathcal{T}} (t_1 < t_2)                                    \\
		 & = \mathfrak{o}(e_p, t_1, t_2) \cdot \mathfrak{c}(e_p, \downarrow s)               \cdot
		\sum_{t_3 \in \mathcal{T}} \prod_{t_2 \in \mathcal{T}} (t_1 < t_2)                                    \\
		 & = \mathfrak{o}(e_p, t_1, t_2) \cdot \mathfrak{c}(e_p, \uparrow (-s))  \cdot
		\sum_{t_3 \in \mathcal{T}} \prod_{t_2 \in \mathcal{T}} (t_1 < t_2)         \text{~\ref{ax:cessation}} \\
		 & \leq \mathfrak{h}(-(-s), t_2, t_3) \hspace{0.31 \textwidth}   \text{~\ref{ax:transition}}          \\
		 & \leq  \mathfrak{h}(s, t_2, t_3).  \hspace{0.27 \textwidth }  \text{Theorem~\ref{thm:arrow_flip}}
	\end{align*}
\end{proof}

A sufficient condition for onsets is that a state does not hold, followed by its complement. We can capture this condition by the following restriction.

\begin{equation}\label{ax:transition_2}
	\prod_{\substack{s \in \mathcal{S} \\ t_1, t_2, t_3 \in \mathcal{T}}}- \mathfrak{h}(s, t_1, t_2) \cdot \mathfrak{h}(s, t_2, t_3) \leq
	\sum_{e_p \in \mathcal{ET}_p} \mathfrak{o}(e_p, t_2, t_2) \cdot \mathfrak{c}(e_p, \uparrow s).
\end{equation}

We can prove a similar theorem for the cessation of a state.

\begin{theorem}\label{thm:cessation_3}
	\begin{equation}
		\prod_{\substack{s \in \mathcal{S} \\ t_1, t_2, t_3 \in \mathcal{T}}}    \mathfrak{h}(s, t_1, t_2) \cdot - \mathfrak{h}(s, t_2, t_3) \leq
		\sum_{e_p \in \mathcal{ET}_p} \mathfrak{o}(e_p, t_2, t_2) \cdot \mathfrak{c}(e_p, \downarrow s).
	\end{equation}

\end{theorem}

\begin{proof}
	\begin{align*}
		\prod_{\substack{s \in \mathcal{S}                                                                                                                                    \\ t_1, t_2, t_3 \in \mathcal{T}}}\mathfrak{h}(s, t_1, t_2) \cdot - \mathfrak{h}(s, t_2, t_3)   \\
		 & =\mathfrak{h}(s, t_1, t_2) \cdot - \mathfrak{h}(s, t_2, t_3)                                                                                                       \\
		 & = \mathfrak{h}(-- s, t_1, t_2) \cdot - \mathfrak{h}(s, t_2, t_3)                                                                                                   \\
		 & = - \mathfrak{h}(- s, t_1, t_2) \cdot - \mathfrak{h}(s, t_2, t_3)                                                                                                  \\
		 & = - \mathfrak{h}(-s, t_1, t_2)   \cdot \mathfrak{h}(-s, t_2, t_3)                                                                                                  \\
		 & \leq \sum_{e_p \in \mathcal{ET}_p} \mathfrak{o}(e_p, t_2, t_2) \cdot \mathfrak{c}(e_p, \uparrow - s) \hspace{0.06 \linewidth} \text{~\ref{ax:transition_2}}        \\
		 & = \sum_{e_p \in \mathcal{ET}_p}\mathfrak{o}(e_p, t_2, t_2) \cdot \mathfrak{c}(e_p, \downarrow s) \hspace{0.06 \linewidth}  \ \ \text{Theorem~\ref{thm:arrow_flip}}
	\end{align*}
\end{proof}

We can add more restrictions based on the type of state. For example, if you are considering
a $\textsc{Perm}$ state, we can prove that no other onset can happen once an onset happens.

\begin{theorem}
	\begin{equation}
		\prod_{\substack{e_{p1} \in \mathcal{ET}_p \\ t_1 \in \mathcal{T}}} \mathfrak{o}(e_{p_1}, t_1, t_1) \cdot \mathfrak{c}(e_{p1}, \uparrow s^{p}) \cdot
		\prod_{\substack{e_{p_2} \in \mathcal{E}_p \\ t_2 \in \mathcal{T}}}[(t_1 < t_2) \cdot \mathfrak{o}(e_{p_2}, t_2, t_2) ]
		\leq - \mathfrak{c}(e_{p_2}, \uparrow s^{p})
	\end{equation}
\end{theorem}

Similarly, we can do the mirror for $\textsc{Co-Perm}$ states.

\begin{theorem}
	\begin{equation}
		\prod_{\substack{e_{p1} \in \mathcal{ET}_p \\ t_1 \in \mathcal{T}}} \mathfrak{o}(e_{p_1}, t_1, t_1) \cdot \mathfrak{c}(e_{p1}, \downarrow s^{cp}) \cdot
		\prod_{\substack{e_{p_2} \in \mathcal{E}_p \\ t_2 \in \mathcal{T}}}[(t_1 < t_2) \cdot \mathfrak{o}(e_{p_2}, t_2, t_2) ]
		\leq - \mathfrak{c}(e_{p_2}, \downarrow s^{cp})
	\end{equation}
\end{theorem}

For \textsc{Temp} states, the pattern of occurrence of onsets and cessations exhibit a more complex temporal
structure: any two onsets (cessations) are separated by at least one cessation (onset). Note that this
may only apply to \textsc{Temp} states since they are the only sort of state that can start or cease more
than once.

\begin{theorem}
	\begin{equation}
		\begin{gathered}
			\prod_{\substack{e_p \in \mathcal{ET}_p \\ t_1 \in \mathcal{T} \\ s \in \mathcal{S}}}\mathfrak{o}(e_{p_1}, t_1, t_1) \cdot \mathfrak{c}(e_{p_1}, \uparrow s^{t}) \cdot \mathfrak{o}(e_{p_2}, t_2, t_2) \mathfrak{c}(e_{p_2}, \uparrow s^{t}) \\
			\cdot (t_1 < t_2) \leq \sum_{\substack{e_{p_2} \in \mathcal{E}_p \\ t_3 \in \mathcal{T}}}[(t_1 < t_3) \cdot (t_3 < t_2) \cdot \mathfrak{o}(e_{p_2}, t_3, t_3)
				\cdot \mathfrak{c}(e_{p_2}, \downarrow s^{t}) )]
		\end{gathered}
	\end{equation}
\end{theorem}

\begin{theorem}
	\begin{equation}
		\begin{gathered}
			\prod_{\substack{e_p \in \mathcal{ET}_p \\ t_1 \in \mathcal{T} \\ s \in \mathcal{S}}}\mathfrak{o}(e_{p_1}, t_1, t_1) \cdot \mathfrak{c}(e_{p_1}, \downarrow s^{t}) \cdot
			\mathfrak{o}(e_{p_2}, t_2, t_2) \mathfrak{c}(e_{p_2}, \downarrow s^{t}) \\
			\cdot (t_1 < t_2) \leq \sum_{\substack{e_{p_2} \in \mathcal{E}_p \\ t_3 \in \mathcal{T}}}[(t_1 < t_3) \cdot (t_3 < t_2) \cdot \mathfrak{o}(e_{p_2}, t_3, t_3)
				\cdot \mathfrak{c}(e_{p_2}, \uparrow s^{t}) )]
		\end{gathered}
	\end{equation}
\end{theorem}

\section{Temporal Closure}
A durative event corresponds to some temporary state starting to hold, holding for a while, and
then ceasing to hold.
The closure maps an onset-cessation pair of some state to a durative event of $s$ maximally holding for a while.

To formalize the above intuition, we introduce the following abbreviations.
\begin{itemize}
	\item \[\textsc{NoOcc}(ec, t_1, t_2) =_{\text{def}}\]
	      \[ -\sum_{\substack{{e \in \mathcal{ET}} \\ t_3, t_4 \in \mathcal{T}}}
		      [\mathfrak{c}(e, ec) \cdot (t_1 < t_3) \cdot (t_4 < t_2) \cdot \mathfrak{o}(e, t_3, t_4)]\]
	\item \[
		      \textsc{OCpair}(e_1, e_2, s, t_1, t_2) =_{\text{def}}
	      \]
	      \[
		      \mathfrak{c}(e_1, \uparrow s) \cdot \mathfrak{c}(e_2, \downarrow s) \cdot \mathfrak{o}(e_1, t_1, t_1) \cdot \mathfrak{o}(e_2, t_2, t_2) \cdot
		      (t_1 < t_2) \cdot \textsc{NoOcc}(\downarrow s, t_1, t_2)
	      \]
\end{itemize}

Let's introduce the $\textsc{Clos}$ relation

$$\textsc{Clos} : \sigma_\mathcal{ET} \to \sigma_\mathcal{ET} \to \sigma_\mathcal{ET}$$

$$\mathfrak{cl} : \mathcal{ET}_p \times \mathcal{ET}_p \to^{\text{Onto}} \mathcal{ET}_d$$

\[ \textlbrackdbl \textlbrackdbl\textsc{Clos}(\tau_1, \tau_2)\textrbrackdbl\textrbrackdbl^{\mathcal{V}} =
	\mathfrak{cl}(\textlbrackdbl \textlbrackdbl\tau_1\textrbrackdbl \textrbrackdbl^{\mathcal{V}},
	\textlbrackdbl \textlbrackdbl \tau_3\textrbrackdbl \textrbrackdbl^{\mathcal{V}})\]

we need to restrict  $\mathfrak{cl}$  so that it only applies to onset-cessation pairs\footnote{We don't have a special sort for onset-cessation.}.

\begin{equation}\label{ax:onset_cessation_pair}
	\prod_{\substack{e_{p_1}, e_{p_2} \in \mathcal{ET}_p \\ t_1, t_2 \in \mathcal{T}}} \mathfrak{o}(\mathfrak{cl}(e_{p1}, e_{p2}), t_1, t_2) =
	\sum_{\substack{s \in \mathcal{S}}}[ \textsc{OCpair}(e_{p_1}, e_{p_2}, s, t_1, t_2]
\end{equation}

Using~\ref{ax:occurs_once} and~\ref{ax:onset_cessation_pair} and the fact that $\mathfrak{cl}$ is onto, we can prove the following theorem.

\begin{theorem}\label{thm:durative_events_temporal_pattern}
	\begin{equation}
		\begin{gathered}
			\sum_{e_d \in \mathcal{ET}_d}[\mathfrak{o}(e_d, t_1, t_2)] \leq  \\
			\sum_{\substack{e_{p_1}, e_{p_2} \in \mathcal{E}_p \\ t_3, t_4 \in \mathcal{T} \\ s \in \mathcal{S}}}[
				\textsc{OCpair}(e_{p_1}, e_{p_2}, s, t_3, t_4)
			]
		\end{gathered}
	\end{equation}
\end{theorem}

This theorem says that durative events are associated with a temporal pattern of onset-cessation pairs of punctual events.\footnote{A durative event is
	made up of a boundary, an onset-cessation pair, and an interior homogeneously filled with a state.}

\begin{proof}
	\begin{align*}
		\sum_{e_d \in \mathcal{ET}_d}[\mathfrak{o}(e_d, t_1, t_2)]                                                 \\
		 & = \mathfrak{o}(e_d, t_1, t_2)                                                                           \\
		 & = \mathfrak{o}(\mathfrak{cl}(e_{p1}, e_{p2}), t_1, t_2) \ \text{Given} \ \mathfrak{cl} \ \text{is onto} \\
		 & = \sum_{\substack{s \in \mathcal{S}                                                                     \\ t_3, t_4 \in \mathcal{T}}}[\textsc{OCpair}(e_{p1}, e_{p2}, s, t_3, t_4)] \ \text{Given~\ref{ax:onset_cessation_pair}}
	\end{align*}
\end{proof}

\section{Packaging and Grinding}
We know about a given durative event that is an event of some state holding for a while. For example, consider the following sentence.

\begin{itemize}
	\item I was fasting from sunrise to sunset.
	\item For a while, I was writing my bachelor's thesis.
\end{itemize}

Both of these sentences refer to some state holding for a \textit{specific} or \textit{unspecific} but \textit{bounded} amount of time. We can package this information into a durative event.

$$\textsc{po}\footnote{Galton makes this particular choice of name for the operator to allude to the Russian prefix po- which has
		the same aspectual effect as the above-introduced \cite{Galton1984-GALTLO}} : \sigma_\mathcal{S} \to \sigma_\mathcal{EC}$$

The $\textsc{po}$ takes a state and returns the event category of that state, maximally holding for a while.
$$\mathfrak{po} : \textsc{Temp} \to \mathcal{EC}_d$$

\[
	\textlbrackdbl\textlbrackdbl \textsc{po}(\tau) \textrbrackdbl \textrbrackdbl ^{\mathcal{V}} =
	\mathfrak{po}(\textlbrackdbl \textlbrackdbl\tau\textrbrackdbl\textrbrackdbl^{\mathcal{V}})
\]

\begin{equation}\label{ax:po}
	\prod_{\substack{e_{p_1}, e_{p_2} \in \mathcal{ET}_p \\ t_1, t_2 \in \mathcal{T} \\ s \in \mathcal{S}}}\textsc{OCpair}(e_{p_1}, e_{p_2}, s, t_1, t_2) = \mathfrak{c}(\mathfrak{cl}(e_{p_1}, e_{p_2}), \mathfrak{po}(s).
\end{equation}

Let us introduce the $\textsc{Prog}$ function as a universal grinder for event categories.
\begin{equation}
	\textsc{prog} : \sigma_\mathcal{EC} \to \sigma_S.
\end{equation}

$$\mathfrak{prog} : \mathcal{EC}_d \to \textsc{Temp}$$
\begin{center}
	\textlbrackdbl $\textsc{Prog}(\tau)$\textrbrackdbl$^{\mathcal{V}}$
	= $\mathfrak{prog}$(\textlbrackdbl $\tau$\textrbrackdbl$^{\mathcal{V}}$)
\end{center}

The $\textsc{Prog}$ function takes an event category and returns the state that holds when that event category is in progress.
\begin{equation}\label{ax:prog}
	\prod_{\substack{e_{p_1}, e_{p_2} \in \mathcal{ET}_d \\ t_1, t_2 \in \mathcal{T}\\ s \in \mathcal{S} \\ ec_d \in \mathcal{EC}_d}}
	[
	\textsc{OCpair}(e_{p_1}, e_{p_2}, s, t_1, t_2 \cdot \mathfrak{c}(\mathfrak{cl} (e_{p_1}, e_{p2}), ec_d)
	] \leq  \mathfrak{h}(\mathfrak{prog}(ec_d), t_1, t_1)
\end{equation}

~\ref{ax:prog} requires the progressive state to hold over the interior of an event; it is silent, not
only about the boundaries but also the exterior of the event. The domain of $\textsc{Prog}$ is the event categories, not the events.
Two distinct events $e_{d1}, e_{d2}$ belonging to the same category $ec_d$ ay overlap in time, giving rise to a continuous
stretch of time; for example, suppose that \textit{Walking} is a durative event category and that I walked from 10:00 to 11:00 and
my friend walked from 10:30 to 11:30. The state $\textsc{Prog}(\textit{Walking})$ holds over the entire stretch of time from 10:00 to 11:30.

% \section{Telicity}
% An event is \textit{telic} if it has a built-in \textit{definite} ending state, otherwise it is \textit{atelic}.\footnote{Vendler has a similar notion of telicity, he defines  this ending state as \textit{climax} \cite{vendler1957verbs}.}
% For example, the following sentences are examples of telic and atelic events, respectively.

% \begin{itemize}
% 	\item He drew.
% 	\item He drew a circle.
% \end{itemize}

% \subsection{Completion}
% Consider the following scenario, in the middle of the event of drawing a circle, the agent stops drawing. The event of drawing a circle is not completed.
% The event of drawing a circle is completed only when the agent stops drawing and the circle is drawn. We can define the completion of an event as follows.
% \begin{equation}
% 	\textsc{complete} : \sigma_\mathcal{ET} \to \sigma_S
% \end{equation}
% $$\mathfrak{complete} : \mathcal{ET} \to \textsc{Prem}\footnote{Although there could be incomplete events, once complete, an event is always complete.}$$
% \begin{center}
% 	\textlbrackdbl $\textsc{Complete}(\tau)$\textrbrackdbl$^{\mathcal{V}}$
% 	= $\mathfrak{complete}$(\textlbrackdbl $\tau$\textrbrackdbl$^{\mathcal{V}}$)
% \end{center}

% An event cannot be completed unless it have ocurred.

% \begin{equation}
% 	\mathfrak{h}(\mathfrak{complete}(e), t_1, t_2) \leq \sum_{t_3, t_4 \in \mathcal{T}}[\mathfrak{o}(e, t_3, t_4) \cdot (t_3 < t_2)]
% \end{equation}

% \subsection{Events Unfolding}
% We usually comprehend events before they occur, For example, planning to work on my bachelor thesis,
% I think of the details of the project; what topic to choose, whom prof to ask for supervision, etc.
% I have a mental image of the event of working on my bachelor thesis awaiting to be ocurred.
% As time passes, I start working on my bachelor thesis, and the event of working on my bachelor thesis starts unfolding.

% What then is a state that characterizes the interior of the whole event? One possible candidate is the progressive state associated with the working-on-bachelor-thesis event category.
% Although this state holds throughout the period of the project, it does not exactly capture what I have in mind during any particular project.
% For the particular bachelor thesis project I worked on, there is a unique state that I got into as I started my project, a state that held throughout the whole event and that never held before, nor will it hold after, the event.

% Formally, there's a \textit{ono-to-one} mapping from durative to their \textit{interiors}.
% \begin{equation}
% 	\textsc{Int} : \sigma_\mathcal{ET} \to \sigma_\mathcal{S}
% \end{equation}

% $$\mathfrak{int} : \mathcal{ET}_{d} \to \textsc{Temp}$$
% \begin{center}
% 	\textlbrackdbl $\textsc{Int}(\tau)$\textrbrackdbl$^{\mathcal{V}}$
% 	= $\mathfrak{int}$(\textlbrackdbl $\tau$\textrbrackdbl$^{\mathcal{V}}$)
% \end{center}

% The following property establishes a basic relation between an event and the state $\mathfrak{int}(s)$
% \begin{equation}\label{ax:int}
% 	\prod_{e \in \mathcal{ET}}\mathfrak{c}(e, \mathfrak{po}(\mathfrak{int}(e))) = \top.
% \end{equation}

% Given~\ref{ax:int}(with~\ref{ax:occurs_once},~\ref{ax:onset_cessation_pair}, and~\ref{ax:po} and the fact that $\mathfrak{cl}$ is onto), we can conclude that
% whenever a durative event $e_d$ occurs, the state $\mathfrak{int}(e_d)$ fills its interior.

% An important property of the interior state is that it is, if they hold, they hold for only a bounded period of time.

% \begin{equation}\label{ax:int_bounded}
% 	\begin{gathered}
% 		\prod_{e_d \in \mathcal{E}^{d}} (\sum_{\substack{t_1, t_2 \in \mathcal{T} \\ e_{p1} \in \mathcal{E}^p}}[\mathfrak{c}(e_{p1}, \uparrow \mathfrak{int}(e_d))
% 		\cdot \mathfrak{o}(e_{p1}, t_1, t_2))] = \\
% 		\sum_{\substack{t_3, t_4 \in \mathcal{T} \\ e_{p2} \in \mathcal{E}^p}}[\mathfrak{c}(e_{p2}, \downarrow \mathfrak{int}(e_d))
% 			\cdot \mathfrak{o}(e_{p2}, t_3, t_4))])
% 	\end{gathered}
% \end{equation}

% Another important thing to add is that the interior states only hold once.

% \begin{equation}\label{ax:int_once}
% 	\begin{gathered}
% 		\prod_{\substack{e_d, e_{d_1}, e_{d_2} \in \mathcal{ET}_d \\ t_1, t_2, t_3, t_4 \in \mathcal{T}}}
% 		\mathfrak{c}(e_{d_1}, \mathfrak{po}(\mathfrak{int}(e_d))) \cdot \mathfrak{c}(e_{d_2}, \mathfrak{po}(\mathfrak{int}(e_d))) \cdot \\
% 		\mathfrak{o}(e_{d_1}, t_1, t_2) \cdot \mathfrak{o}(e_{d_2}, t_3, t_4) = \bot \quad \text{if} \quad e_{d_1} \neq e_{d_2}.
% 	\end{gathered}
% \end{equation}

% Over any interval, if the state $\mathfrak{int}(e_d)$ holds, then the progressive state associated with the event category of $e_d$ holds.

% \begin{equation}\label{ax:int_progressive}
% 	\begin{gathered}
% 		\prod_{\substack{e_d \in \mathcal{ET}_d \\ t_1, t_2 \in \mathcal{T} \\ ec_d \in \mathcal{EC}_d}}
% 		\mathfrak{c}(e_d, ec_d) \cdot \mathfrak{h}(\mathfrak{int}(e_d), t_1, t_2) \leq \\
% 		\mathfrak{h}(\mathfrak{prog}(ec_d), t_1, t_2)
% 	\end{gathered}
% \end{equation}

% The following property expresses an important relation between an event $e_d$ and temporal closures of $\mathfrak{int}(e_d)$.

% \begin{equation}
% 	\begin{gathered}
% 		\prod_{\substack{e_{p_1}, e_{p_2} \in \mathcal{ET}_p \\ t_1, t_2 \in \mathcal{T} \\ e_d \in \mathcal{ET}_d}}
% 		\mathfrak{o}(\mathfrak{clos}(e_{p_1}, e_{p_2}), t_1. t_2) \cdot \textsc{OCpair}(e_{p_1}, e_{p_2}, \mathfrak{int}(e_d), t_1, t_2) \leq \\
% 		\sum_{t^\prime_1, t^\prime_2 \in \mathcal{T}}[\mathfrak{h}(\mathfrak{complete}(e_d), t^\prime_1, t^\prime_2)]
% 	\end{gathered}
% \end{equation}