\chapter{Events}

Up until now, we have only considered \textit{states} in our representation using $Log_A S$. However, states alone cannot represent certain situations, such as the distinction between the following two sentences:

\begin{enumerate}[label=(\arabic*)]
    \item I ate an apple.
    \item I was eating an apple.
\end{enumerate}

\begin{equation}
    \textsc{HoldsAt}(\textsc{Ate}(I, \textsc{Apple}), t_1, t_2)
    \label{eq:ate}
\end{equation}
Sentence (1) uses the perfective aspect and presents the situation as a complete, finished event, while sentence (2) uses the imperfective aspect and focuses on the ongoing nature of the situation without indicating whether it has ended.

To represent events, we need to extend our language. Equation~\ref{eq:ate} represents a state, but we need to represent an event, so we need to introduce events into our language
which we will call $Log_A T$.

\section{Tokens and Categories of Events}
\subsection{Event Tokens}

In contrast to states, event tokens do not hold homogeneously over time.
Instead, they occur as complete wholes.
To capture this notion, we introduce a new sort $\mathcal{ET}$ for event tokens and a new function $\textsc{Occurs}$ that relates events to the time they occur.

\begin{defn}
    The function $\textsc{Occurs}: \mathcal{ET} \times \mathcal{T} \times \mathcal{T} \to \textsc{Eter}$ maps an event token,
    a starting time, and an ending time to an eternal state.
\end{defn}

\begin{defn}
    \textlbrackdbl $\textsc{Occurs}(\tau_1, \tau_2, \tau_3)$\textrbrackdbl$^{\mathcal{V}}$ = $\mathfrak{o}$(\textlbrackdbl $\tau_1$\textrbrackdbl$^{\mathcal{V}}$, \textlbrackdbl $\tau_2$\textrbrackdbl$^{\mathcal{V}}$, \textlbrackdbl$\tau_3$\textrbrackdbl$^{\mathcal{V}}$)
\end{defn}

The occurrence of an event is eternal, meaning that if an event occurred between two time points, it will always have occurred between those two time points. Additionally, each occurrence is unique, meaning that an event token occurs only once. This is captured by the following axiom:

To ensure that an event token occurs only once between any two time points, we add the following axiom.

\begin{axiom}
    \begin{equation}
        \mathfrak{o}(e, t_1, t_2) \cdot \mathfrak{o}(e, t_3, t_4) = \bot  \ if \ t_1 \neq t_3 \ or \  t_2 \neq t_4
    \end{equation}
    \label{ax:occurs_once}
\end{axiom}

Intuitively, $\mathfrak{o}(e_1, t_1, t_2)$ means that the event token $e$ occurs at some time between $t_1$ and $t_2$. We also add an axiom to ensure that an event token occurring between $t_1$ and $t_2$ is followed by $t_2$:

\begin{axiom}
    \begin{equation}
        \mathfrak{o}(e, t_1, t_2) \cdot (t_2 < t_1) = \bot
    \end{equation}
\end{axiom}

\begin{theorem} Events are heterogeneous.
    \begin{equation}
        \mathfrak{o}(e, t_1, t_2) \leq \prod_{t_3, t_4 \in \mathcal{T}} [(t_3 < t_2) \cdot (t_4 < t_1)
            \leq - \mathfrak{o}(e, t_3, t_4)].
        \label{eq:events_heterogeneous}
    \end{equation}
    \begin{equation}
        \mathfrak{o}(e, t_1, t_2) \leq \prod_{t_3, t_4 \in \mathcal{T}} [(t_3 < t_1) \cdot (t_2 < t_4)
            \leq - \mathfrak{o}(e, t_3, t_4)].
        \label{eq:events_heterogeneous2}
    \end{equation}
\end{theorem}

Both Equation~\ref{eq:events_heterogeneous} and Equation~\ref{eq:events_heterogeneous2} follows from the 
axiom~\ref{ax:occurs_once}.

That means that if an event occurs over some interval, it cannot occur over any super or sub-interval.
We think of the interval as the smallest part the event can fit in.\footnote{Allen \cite{allen1984towards} defines events in a similar way, if an event occurs over some interval, it cannot occur over any sub-interval.
}

\subsection{Event Categories}
We can easily recognize that some event token is \textit{eating an apple}. So we will add this notion of \textit{category}\footnote{VEL has a similar notion called \textit{event radicals} \cite{bennett2001unifying}} to our language.

Let $\mathcal{EC}$ be the set of event categories. We will use the following function to relate event tokens to event categories.

\[
    \textsc{Cat}: \mathcal{ET} \times \mathcal{EC} \to \mathcal{S}
\]

Unlike $\textsc{Occurs}$, $\textsc{Cat}$ does not always return eternal states. 

TODO: add justification for this.
%%TODO: add justification for this.


