\chapter*{Future Work}
\addcontentsline{toc}{chapter}{Future Work}

\section{Punctual Events}
\label{sec:punctual_events}

Punctual events are events that occur at a specific point in time. They are
either onsets or cessations of a state. However, at any given point in time,
there can be a punctual event for both the onset and the cessation of a state.
So do really need them in the language? Or can we just use time points to
represent them? This is a question that needs to be investigated further.

\section{Telicity}
\label{sec:telecity}

Telicity is a property of events that is related to the notion of completion.
An event is telic if it has a natural endpoint. For example, the event of
reading a book is telic because it has a natural endpoint, which is finishing
reading the book. On the other hand, the event of running is atelic because it
does not have a natural endpoint. The event of running can be terminated at any
point in time. The event of reading a book, however, cannot be terminated
arbitrarily. It can only be terminated when the book is finished.

The notion of telicity is important and needs to be investigated and integrated
into the language.