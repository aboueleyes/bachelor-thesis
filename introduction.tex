\chapter{Introduction}
The concept of time has long been a subject of philosophical inquiry, and computer science has increasingly shown interest in this area.
Philosophers and logicians have studied the nature of time, including questions of whether it is \textit{discrete} or \textit{continuous}, \textit{bounded} or \textit{unbounded}, and \textit{linear} or \textit{cyclical}.
Despite ongoing debates, people generally have a common-sense understanding of time that enables them to navigate the world and communicate effectively.

Psychologists and philosophers distinguish between three categories of time: \textit{natural}, \textit{conventional}, and \textit{logical}.

\textit{Natural time} is based on the occurrence of natural phenomena, such as day and night and the seasons, while \textit{conventional time} is based on social conventions, such as the measurement of seconds, minutes, and hours.
\textit{Logical time} refers to the mathematical structure of time and its operations.

Temporal knowledge and reasoning are important in many fields, including computer science, where they are essential for designing information systems, verifying programs, and developing artificial intelligence applications. 

In discussions of aspectual phenomena, \textit{events} are another prominent type of entity that can be attributed to temporal incidence. 
Unlike \textit{states}, events \cite{vendler1957verbs} occur and are bounded in time, having a distinct beginning and end. Events are a fundamental concept in many areas of artificial intelligence and related fields, such as the situation calculus and the event calculus.
Events are also recognized in other domains, such as in linguistics, where they are seen as the basic building blocks of narrative discourse.

An investigation of events is therefore crucial in understanding various cognitive and linguistic phenomena, such as the representation of causation, the interpretation of tense and aspect, and the construction of narratives.
%% cite smith(logAS) instead of vendler.

% \textbf{TALK ABOUT THE $LOG_AS$}

% \section{Thesis Chapters}
% TO BE FILLED IN.
